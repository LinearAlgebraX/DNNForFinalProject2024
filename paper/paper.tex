\documentclass[12pt]{report}
\usepackage{gnuplottex}
\usepackage{csvsimple}
\usepackage{subcaption}
\usepackage{amsmath}
\usepackage{multirow}
\usepackage[top=1cm,bottom=1cm,left=1cm,right=1cm]{geometry}
\usepackage[justification=centering]{caption}


\title{Overactivation - Verifying and testing backdoors in deep neural networks}
\author{Xin Wu}

\begin{document}
\maketitle

\section*{Abstract}
-- is not the final version

    With the widespread integration of deep neural networks (DNNs) into contemporary societal 
    applications, concerns are mounting over their susceptibility to malicious attacks, one such of it is 
    backdoor attacks. Instances such as Tesla's autonomous driving accidents and the prevalence of 
    adversarial attacks on AI image generators underscore the urgent need for robust security measures 
    within DNNs.

    This paper proposes a novel backdoor testing approach that leverages the phenomenon of neuron 
    overactivation within DNNs to directly detect the presence of backdoors in models. To evaluate 
    the proposed method, extensive testing and research were conducted on popular datasets and across 
    different models. Experimental results demonstrate the effectiveness of the approach in revealing 
    variations between various model structures and datasets.

    In conclusion, this study introduces a valuable tool to mitigate the risk of backdoor attacks in 
    DNNs, providing a reliable testing method to enhance the overall security of these systems. The 
    effectiveness demonstrated in discerning potential threats highlights the potential of this method 
    in fortifying artificial intelligence systems against malicious manipulation.

\tableofcontents{}

\chapter{Introduction}
    Deep neural networks (DNNs) have revolutionized image recognition in various fields,
    including autonomous driving\cite{A1}, medical diagnostics\cite{B1}, and security surveillance\cite{A2}. Their 
    remarkable success, however, is shadowed by their vulnerability to adversarial attacks, 
    which pose significant threats to their reliability and security. Among these adversarial 
    threats, backdoor attacks have emerged as a particularly challenge.

    Backdoor attacks involve the manipulation of DNN behavior by malicious actors through the 
    injection of images with triggers into the training dataset. Once deployed, these triggers 
    can activate specific behaviors or misclassifications, compromising the integrity and 
    trustworthiness of the entire system. Detecting and mitigating such attacks are paramount 
    for ensuring the robustness and trustworthiness of DNN-based image recognition systems.

\section{Aims}
    Related work: One of the existing model backdoor detection tools, NeuralCleanse, operates under 
    the assumption that smaller modifications will result in misclassification by the model. Another 
    detection method, STRIP, requires prior knowledge of trigger-related information to ascertain 
    whether a model has been injected with a backdoor. This project aims to explore a novel method 
    that offers greater flexibility in usage conditions compared to existing tools and methods. 
    
    Additionally, it aims to provide insights that could inspire advancements in the field of 
    backdoor detection to some extent.
\section{Structure of this Project}


\chapter{Background}
\section{literature review}


\begin{table*}[t]
    \centering
    \caption{The information about each models\\(for CIFAR100 the attack have some bugs So I need retraining it later.)}
    \label{table1}
    \begin{tabular}{|c|c|c|c|c|}
        \hline
        Models & epoch & Clean Accuracy & ASR & Architecture \\
        \hline
        MNIST & 5 & 99.10$\%$ & 99.99$\%$ &2 conv$\times$1 dense \\
        \hline
        GTSRB & 5 & 96.255$\%$ & 91.100$\%$ &5 conv \\
        \hline
        FashionMNIST & 5 & 92.81$\%$ & 99.98$\%$ & 2 conv$\times$1 dense \\
        \hline
        CIFAR10 & 50 & 91.44$\%$ & 99.99$\%$ & ResNet18 \\
        \hline
        CIFAR100 & 100 & 73.84$\%$ & 100$\%$ & ResNet50 \\
        \hline
    
        
    \end{tabular}
    
\end{table*}


\begin{table*}[t]
    \centering
    \caption{Below showed the Activation index($S_A$ means the set of maximum value which produced by the feature map from 1st Relu() for each noise images) for each dataset in different epoch. Totally have 200 noise images.}
    \label{table2}
    \begin{tabular}{|c|c|c|c|c|c|c|c|c|}
    \hline
    \multicolumn{2}{|c|}{\multirow{2}{*}{Model}} & \multicolumn{3}{c|}{MNIST}&\multicolumn{4}{c|}{GTSRB}\\
    \cline{3-9}
    
    \multicolumn{2}{|c|}{} & 5 epoch & 10 epoch & 20 epoch & 10 epoch & 20 epoch & 30 epoch & 50 epoch \\
    \hline

    % Start the main table

    \multirow{3}{*}{max($S_A$)} & \textbf{Backdoor} &3.9615 &3.5404 &5.0043 &4.7136 &4.3301 &4.611 &3.5434 \\
    \cline{2-9}
    
    \multirow{3}{*}{} & \textbf{Clean} &3.2132 &3.9103 &5.1311 &4.345 &3.3928 &3.5097	&3.2779\\
    \cline{2-9}
    
    \multirow{3}{*}{} & \textbf{B/C} &\textbf{1.2328} &\textbf{0.9054} &\textbf{0.9752} &\textbf{1.0848} &\textbf{1.2762}	&\textbf{1.3138}&\textbf{1.081} \\
    \hline

    \multirow{3}{*}{mean($S_A$)} & \textbf{Backdoor} &3.2140 &2.6816 &3.6324 &3.6496 &2.9364 &3.1777 &2.3143 \\
    \cline{2-9}
    
    \multirow{3}{*}{} & \textbf{Clean} &2.5139	&3.0669 &3.8204 &3.216 &2.5081	&2.0844 &2.3850\\
    \cline{2-9}
    
    \multirow{3}{*}{} & \textbf{B/C} &\textbf{1.2784} &\textbf{0.8743} &\textbf{0.9507} &\textbf{1.1348} &\textbf{1.1708}	&\textbf{1.5245} &\textbf{0.9704} \\
    \hline

    \multicolumn{2}{|c|}{Ratio($1stRelu_p$\textgreater $1stRelu_c$)} &\textbf{199:1}    &\textbf{17:183}	&\textbf{38:162}	&\textbf{163:37}	&\textbf{177:23}	&\textbf{195:5}	&\textbf{109:91}\\ 
    
    \hline
    & & & & & & & & \\
    \hline

    \multicolumn{2}{|c|}{\multirow{2}{*}{Model}} & \multicolumn{3}{c|}{FashionMNIST}&\multicolumn{4}{c|}{CIFAR10}\\
    \cline{3-9}
    
    \multicolumn{2}{|c|}{} & 5 epoch & 10 epoch & 30 epoch & 20 epoch & 30 epoch & 40 epoch & 50 epoch \\
    \hline

    % Start the main table

    \multirow{3}{*}{max($S_A$)} & \textbf{Backdoor} &1.5759 &1.8628 &2.4023 &12.936 &8.8327 &5.6955 &4.2566 \\
    \cline{2-9}
    
    \multirow{3}{*}{} & \textbf{Clean} &1.3684 &1.8138 &2.6442 &9.8593 &7.4869 &6.5439	&4.6565\\
    \cline{2-9}
    
    \multirow{3}{*}{} & \textbf{B/C} &\textbf{1.1516} &\textbf{1.027} &\textbf{0.9085} &\textbf{1.312} &\textbf{1.1798}	&\textbf{0.8704}&\textbf{0.9141} \\
    \hline

    \multirow{3}{*}{mean($S_A$)} & \textbf{Backdoor} &1.313 &1.4639 &2.0127 &9.9769 &6.5749 &4.387 &3.4229 \\
    \cline{2-9}
    
    \multirow{3}{*}{} & \textbf{Clean} &1.2007	&1.4822 &2.2154 &7.7057 &5.2923	&5.2367 &3.6726\\
    \cline{2-9}
    
    \multirow{3}{*}{} & \textbf{B/C} &\textbf{1.0934} &\textbf{0.9877} &\textbf{0.9085} &\textbf{1.2947} &\textbf{1.2423}	&\textbf{0.8377} &\textbf{0.932} \\
    \hline

    \multicolumn{2}{|c|}{Ratio($1stRelu_p$\textgreater $1stRelu_c$)} &\textbf{189:11}    &\textbf{95:105}	&\textbf{12:188}	&\textbf{196:4}	&\textbf{194:6}	&\textbf{11:189}	&\textbf{60:140}\\ 
    \hline
    
    \end{tabular}
    % \label{table_MAP}
    \end{table*}




\bibliographystyle{plain}
\bibliography{refPaper}
\end{document}